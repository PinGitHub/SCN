% secdoc.tex V2.0, 28 June 2010

\documentclass[times]{secauth}

\usepackage{moreverb}

\usepackage[colorlinks,bookmarksopen,bookmarksnumbered,citecolor=red,urlcolor=red]{hyperref}

%\usepackage{multirow}
%\usepackage{multicol}

\usepackage{mcite}
\usepackage{graphicx}
\usepackage{float}
\usepackage{leftidx}
\usepackage{amssymb}
\usepackage[caption=false]{subfig}
\usepackage{bm}
\usepackage{array}
\usepackage{enumerate}
\usepackage{url}
\usepackage{amsmath}
\usepackage{amsthm}
\newtheorem{theorem}{Theorem}[section]
\newtheorem{lemma}[theorem]{Lemma}
%
\usepackage{cite}
%
\theoremstyle{definition}
\newtheorem{definition}[theorem]{Definition}
\theoremstyle{remark}
\newtheorem{remark}[theorem]{Remark}

\newcommand{\tabincell}[2]{\begin{tabular}{@{}#1@{}}#2\end{tabular}}

\newcommand\BibTeX{{\rmfamily B\kern-.05em \textsc{i\kern-.025em b}\kern-.08em
T\kern-.1667em\lower.7ex\hbox{E}\kern-.125emX}}

\def\volumeyear{2010}

\begin{document}

\runningheads{P.~Chen.~Other}{An EF-HIBOOS Scheme for Privacy Protection of PCS}

\articletype{RESEARCH ARTICLE}

\title{An Escrow-Free Online/Offline HIBS Scheme for Privacy Protection of People-Centric Sensing}

\author{Peixin Chen$^1$\corrauth, Jinshu Su$^{2, 1}$, Xiaofeng Wang$^1$, Baokang Zhao$^1$ and Ilsun You$^3$}

\address{$^1$ College of Computer, National University of Defense Technology, Changsha 410073, China\\
$^2$ State Key Laboratory of High Performance Computing, National University of Defense Technology, Changsha 410073, China\\
$^3$ Department of Information Security Engineering, Soonchunhyang University, Asan-si, Republic of Korea}

\corraddr{College of Computer, National University of Defense Technology, No. 109 Deya Street, Changsha 410073, Hunan, China.}

\begin{abstract}
People-Centric Sensing (PCS), which collects information closely related to human activity and interactions in societies, is stepping into a flourishing time. 
Along with its great benefits, PCS poses new security challenges such as data integrity, participant privacy. 
Hierarchical Identity-Based Signature (HIBS) scheme can efficiently provide high integrity messaging, secure communication and privacy protection to PCS.
However, key escrow problem and low computation efficiency primarily hinder the adoption of HIBS scheme.
In this paper, we propose an escrow-free online/offline HIBS (EF-HIBOOS) scheme for securing PCS.
By utilizing user-selected-secret signing algorithm and splitting the signing phase into online and offline procedures, our scheme solves the key escrow problem and achieve high signing efficiency.
\end{abstract}

\keywords{Internet of Things; People-Centric Sensing; Hierarchical Identity-Based Signature; Online/Offline Signature; Key Escrow Problem}

\maketitle

\footnotetext[2]{Please ensure that you use the most up to date
class file, available from the SEC Home Page at\\
{\tiny\href{http://www3.interscience.wiley.com/journal/114299116/home}{\texttt{http://www3.interscience.wiley.com/journal/114299116/home}}}}

\section{Introduction} 
With enormous improvements of sensing technology and embed computation, more and more ubiquitous devices are utilized to build a new kind of mobile sensing system, which is referred to as People-Centric Sensing (PCS) system \cite{campbell2008rise}. 
However, the PCS is facing a more serious privacy problem than the traditional wireless sensor network.
Works on protect the privacy of PCS has been proposed \cite{cornelius2008anonysense,shi2010prisense,puttaswamy2010preserving,johnson2007people}. 
Most of the work assume that participant of PCS application are supported from a public key infrastructure (PKI).
Nevertheless, building and operating a PKI are quite burden jobs, which significantly reduce the practicability of the PKI-based scheme.
The Identity-Based Signature (IBS) scheme can be efficiently utilized to build a lightweight PKI that is appropriate for the PCS application.
\par 

IBS is a public key signature scheme which allows a receiver to verify message using the signer's identity as the public key \cite{shamir1985identity}. 
Several IBS and hierarchical IBS (HIBS) schemes \cite{choon2002identity,chow2004secure,gerbush2012dual} have been presented since the idea was proposed. 
However, an HIBS scheme uses Private Key Generators (PKGs) to generate private key for users so that it inevitably leads to the key escrow problem. 
That is, the PKG knows the private keys and thus can unscrupulously sign messages intended for the users \cite{boneh2001identity}. 
Works on relieving and solving key escrow problem have been being proposed \cite{boneh2001identity,kate2010distributed,cao2011sa,zhang2012efficient,yuen2010construct,chen2015efhibs}.
Boneh et al. apply the threshold method to suggest an $(n,t)$ distributed PKG mechanism \cite{boneh2001identity}. 
Kate and Goldberg improve their distributed PKGs model and apply it to three well-known IBE schemes
\cite{kate2010distributed}. 
However, the efficiency concern makes this type of escrow-free model impractically to be adopted.
Yuen et al. focus on the signature scheme and provide a formal model for constructing escrow-free IBS scheme \cite{yuen2010construct}. 
A judge and a Trusted Third Party is required to take part in their escrow-free model.
Zhang et al. propose an escrow-free IBS scheme that unnecessary depends on any judges \cite{zhang2012efficient}. 
Either Yuen's model or Zhang's scheme is only for IBS scheme and cannot be applied to HIBS schemes. 
\par

Besides the key escrow problem, the low computation efficiency of HIBS scheme is another concern in the identity-based cryptography.
Most IBS schemes involve computations including pairings over points on elliptic cure and point multiplications in groups, which might be too costly to be applied in lightweight devices.
Even et al. first introduce the notion of online/offline signatures to reduce the computational cost of signature generation \cite{even1990line}.
An OO signature scheme divides the process of message signing into offline phase and online phase.
The offline phase with most of the heavy computations is performed prior to obtaining the message to be signed.
And the online phase performing the light computations is executed when messages are ready. 
According to the OO signature model, several identity-based online/offline signature scheme have been proposed \cite{liu2011online,yasmin2010authentication,liu2010efficient,kar2014provably}.
Liu et al. propose an online/offline IBS scheme (LJY-IBOOS) for securing wireless sensor network \cite{liu2010efficient}.
Yang et al. illuminate that the LJY-IBOOS scheme is not secure and propose an identity-based online/offline threshold signature scheme, which achieves higher security \cite{yang2013id}.
Both of their schemes are not for hierarchical IBS.
Since each cryptosystem needs to be proved security utilizing formal mathematical methodology, it cannot directly extend a proved-secure IBS scheme to a secure HIBS scheme.
Moreover, a secure HIBS scheme has to solve the domain-PKG collusion problem in addition.
\par

In this paper, we propose an escrow-free online/offline hierarchical identity-based signature (EF-HIBOOS) scheme, which only executes two group element exponentiation operations in online procedure. 
We use a user-selected secret while signing messages so that any bogus signatures can be recognized and the forgery behavior will be blamed. 
We prove that our scheme is adaptive chosen-message and identity existential-forgery (EF-ID-CMA) secure and can efficiently solve the key escrow problem.
\par

\section{Preliminaries}\label{sec-Pre}
In this section, we review some background knowledge, including the bilinear pairing and the complexity assumption used in our proof. 
\subsection{Bilinear Pairing}
Let $G_1$ and $G_2$ be two cyclic multiplicative groups of the same order $p$. 
A map $e: G_1 \times G_1 \rightarrow G_2$ is referred to as a bilinear paring if it has the following properties: 
\begin{enumerate}[1.]
\item Bilinear: $\forall u,v \in G_1, a,b \in \mathbb{Z}_N$, there is $e(u^a, v^b) = e(u,v)^{ab} $;
\item Non-degenerate: $\exists g \in G_1$, s.t. $e(g,g) \neq 1$, where $1$ denotes the identity in $G_2$;
\item Computable: $\forall u, v \in G_1$, there is an efficient algorithm to compute $e(u,v)$.
\end{enumerate}

\subsection{Complexity Assumption}
We prove the security of our scheme based on the CDH assumption. 
The assumption is defined as follows: 
\begin{definition}[CDH Assumption]
Let $G$ be be a cyclic multiplicative group generated by $g$ and $a, b \in \mathbb{Z}_p$.
Given $g, g^a, g^b$, there is no probabilistic polynomial time algorithm $\mathcal{A}$ has a non-negligible advantage to compute the value $g^{ab}$.
\end{definition}

\section{Overview of our escrow free HIBOOS model}\label{sec-overview}
In this section, we firstly introduce the intuition of our solution to the key escrow problem of HIBS. 
Then, we briefly describe the construction of our scheme. 
Finally, we present the full security definition by illuminating the \emph{EF-ID-CMA}, \emph{EKA-ID-CMA} and \emph{EUS-ID-CMA} attack games for our escrow-free approach.

\subsection{Intuition}
In the HIBS scheme, a user private key is generated by a domain PKG. 
Therefore, either the domain PKG or the user can sign a message to obtain a valid signature. 
Since the signature verifier cannot determine the actual signer, two problems should be addressed in those primitive HIBS scheme:
\begin{itemize}
\item \emph{Key abusing problem}. A domain PKG is able to sign messages with the user keys generated by it without being detected;
\item \emph{User slandering problem}. The dishonest user is able to sign a message and slander that the PKG abuses its private key. That is, the undeniable property is missing in the primitive HIBS scheme. 
\end{itemize}
\par 
For the key abusing problem, an intuitive solution is to limit the signing ability of the PKG by a well designed signing algorithm.
Therefore, we use a user-selected secret apart from the private key to generate the signature for a message. 
We also compute a user public parameter with the secret as input.
The user sends the parameter along with the message and signature to the receivers. 
Signature verifying needs to take user parameter and signature as input. 
Since the PKG cannot obtain the user secret, it cannot generate a valid signature with respect to the user parameter. 
However, the PKG can generate a well-formed signature with a fake user parameter. 
Receiver will thereby accept the PKG generated signature-parameter pair. 
To solve such problem, we introduce an Arbitral Party (AP) to keep the users' public parameters.
User publishes its user parameter, and attaches these same parameter to each signature.
A receiver is not constrained to compare the user parameter attaching in the signature with the one publishing in the AP.
Nevertheless, the PKG will be detected and blamed once it abuses a user private key to sign messages. 
Note that, an AP does not keep any confidential contents.
\par
Since PKG is able to generate well-formed signatures with distinct user parameters, a user can slander the PKG by signing messages with randomly picked secret and sending the receiver a corresponding fake user parameter along with the signature. 
To solve this problem, we modify the signing algorithm so that the user can only generate well-formed signatures with regard to the parameter it published. 
After publishes the public parameter, user also needs to ask for a PKG signing factor from the root PKG. 
The root PKG computes the factor with the user parameter as well as the master secret as input and returns the factor to the user.
User is desired to sign messages with the PKG signing factor. 
\par 
Combining these two technology, we can present a full secure escrow-free HIBS model. 
Generally speaking, it can be applied to any secure HIBS schemes by modifying the signing and verifying algorithms.
In this paper, we instantiate an escrow-free HIBS scheme with the above intuition. 
The construction and security proof is described in Section \ref{sec-EFOOHIBS}.

\subsection{Generic Construction}
A primitive hierarchical identity-based signature scheme generally consists of four algorithms: Setup, KeyGen, Signing and Verification.
To achieve the escrow-free property, we add Publish algorithm and Blame algorithm to our HIBS scheme.
The generic construction of our HIBS scheme is as follows:
\vspace{0.1cm}
\\
\textbf{Setup.}
The setup algorithm takes a security parameter as input and outputs the HIBS public parameters. 
\vspace{0.1cm}
\\
\textbf{KeyGen.}
The key generation algorithm takes a secret key and an identity ID as input and outputs the user private key. 
More specifically, the root PKG takes the master secret as input and can generate private keys for any user.
Each domain PKG takes its private key as input and generates private keys for its descendant user.
\vspace{0.1cm}
\\
\textbf{Publish.}
The algorithm takes the user secret as input and outputs the user parameter. User uploads the parameter to the AP and get PKG signing factor from the root PKG.
\vspace{0.1cm}
\\
\textbf{Offline Signing.}
The offline signing algorithm is performed prior to obtaining the message. 
It takes a private key as well as the public parameter as input, and outputs the offline signature.
\\
\textbf{Online Signing.}
The online signing algorithm takes a message, a private key and the offline signature as input, and outputs a final signature.
\vspace{0.1cm}
\\
\textbf{Verification.}
The verification algorithm takes an identity $\mathrm{ID}$, a message $m$ and a signature as input.
If $\sigma$ is valid, outputs 1. 
Otherwise, outputs 0.
\vspace{0.1cm}
\\
\textbf{Blame.}. 
The algorithm takes a message-signature pair $\{m, \sigma\}$  and a user parameter as input.
If $\sigma$ is generated by an honest signer, outputs 0.
Otherwise, outputs 1.

\subsection{Security Model}
We claim that an escrow-free hierarchical identity-based online/offline signature scheme with full security should be uncrackable against three attack games: \emph{EF-ID-CMA}, \emph{EKA-ID-CMA} and \emph{EUS-ID-CMA}. 
\par
Extend the game, for OO signature ...
要说清楚几个安全游戏的区别,包括但不限于:1.B什么时候可以指定sID,什么时候不可以;2.最后的挑战的区别;3.A的权利的区别

\begin{definition}[\emph{EF-ID-CMA} Security]
We say that a hierarchical identity-based signature scheme is secure if no probabilistic polynomial time (\emph{PPT}) adversary $\mathcal{A}$ has a non-negligible advantage against the challenger $\mathcal{C}$ in the above \emph{EF-ID-CMA} game. 
As shorthand, we say that the HIBS scheme is \emph{EF-ID-CMA} secure.
\end{definition}
\par
\begin{definition}[\emph{EKA-ID-CMA} Security]
We say that a hierarchical identity-based signature scheme is secure if no \emph{PPT} adversary $\mathcal{A}$ has a non-negligible advantage against the challenger $\mathcal{C}$ in the above \emph{EKA-ID-CMA} game. 
As shorthand, we say that the HIBS scheme is \emph{EKA-ID-CMA} secure.
\end{definition}
\begin{definition}[\emph{EUS-ID-CMA} Security]
We say that a hierarchical identity-based signature scheme is secure if no \emph{PPT} adversary $\mathcal{A}$ has a non-negligible advantage against the challenger $\mathcal{C}$ in the above \emph{EUS-ID-CMA} game. 
As shorthand, we say that the HIBS scheme is \emph{EUS-ID-CMA} secure.
\end{definition}
\par
Details of the security model can be found in our previous work \cite{anescrowfree2015chen}.


\section{Escrow-free Online/Offline HIBS Scheme}\label{sec-EFOOHIBS}

\subsection{Construction}
Let $K$ be the security parameter given to the setup algorithm, and let $\mathcal{G}$ be a BDH parameter generator.
 %\vspace{0.2cm}
\\
\textbf{Setup.} Given a security parameter, the PKG works as follows:
\begin{enumerate}
\item runs $\mathcal{G}$ on input $K$ to generate multiplicative groups $G_1, G_2$ of same prime order, and a bilinear pairing $\hat{e}: G_1 \times G_1 \rightarrow G_2$;
\item chooses random $\alpha \in \mathbb{Z}^*_p$ and two generators $g, g_2 \in G_1$, computes $g_1 = g^\alpha$;
\item randomly picks $h_1, \ldots, h_\ell \in G_1$;
\item chooses two cryptographic hash functions $H_1: \{0, 1\}^* \times G_1 \rightarrow \mathbb{Z}_p^*$ and $H_2: G_1 \times \{0, 1\}^* \rightarrow G_1$;
\item publishes $Param = \{\hat{e}, g, g_1,  g_2, h_1, \ldots, h_\ell, H_1, \\H_2\}$ as public parameters and keeps $\mathrm{MSK} = g_2^\alpha$ as master secret.
\end{enumerate}
\textbf{KeyGen.} For an input $\mathrm{ID} = \{I_1, \ldots, I_k\}$, the $level_{k-1}$ domain PKG with private key $d_{\mathrm{ID}_{\mid k-1}} = \{d'_0, \ldots, d'_{k-1}\}$ generates the key $d_\mathrm{ID}$ as follows:
\begin{enumerate}
\item picks random $r_k \in \mathbb{Z}_p^*$;
\item set $d_{\mathrm{ID}} = \{d'_0F_k(I_k)^{r_k}, d'_1, \ldots, d'_{k-1}, g^{r_k}\}$,\\ where $F_k(x)$ $ = g_1^xh_k$.
\end{enumerate}
\par
We refer to $g_2^\alpha$ as $d_{\mathrm{ID}_{\mid 0}}$, and the user private key can be presented as 
$d_{\mathrm{ID}} = \{d_0, d_1, \ldots, d_k\} = \{g_2^\alpha \prod^k_{j=1} F_j(I_j)^{r_j},$ $g^{r_1}, \ldots, g^{r_k}\}$.
\vspace{0.2cm}
\\
\textbf{Publish.} In this phase, user publishes a public parameter and gets PKG signing factor from the root PKG. It does the work as follows:
\begin{enumerate}
\item picks $s_{\mathrm{ID}} \in \mathbb{Z}_p^*$ as user secret;
\item computes $g_{\mathrm{ID}} = g^{s_{\mathrm{ID}}}$;
\item publishes $g_{\mathrm{ID}}$ by submitting it to the AP;
\item sends $g_{\mathrm{ID}}$ to the root PKG, the root PKG computes and returns $f^\alpha = H_2(g_\mathrm{ID}, \mathrm{ID})^\alpha$. 
\end{enumerate}
Although the root PKG has to computes the signing factor for all users, it brings in little overhead because of the following reasons:
\begin{itemize}
\item it has not to authenticate the users before computing the signing factors for them;
\item it has not to maintain a secure channel to transmit the signing factor;
\item it only needs to do an exponentiation computation for each user.
\end{itemize}
\textbf{Offline Signing.} 
During this phase, the signer computes the followings:
To sign a message $m \in \{0, 1\}^*$ with respect to identity $\mathrm{ID} = \{I_1, \ldots, I_k\}$, user takes the private key $d_{\mathrm{ID}} = \{d_0, d_1, \ldots, d_k\}$, secret $s_{\mathrm{ID}}$ and PKG signing factor $f^\alpha$ as input, running the signing algorithm as follows:
\begin{enumerate}
\item picks a random $s \in \mathbb{Z}_p^*$ and computes $x = g_2^s$;
\item for $j = 1, \ldots, k$, computes $y_j = d_j^{s}$;
\item computes $d_0^{s}$;
\item computes $f = H_2(g_\mathrm{ID}, \mathrm{ID})$;
\item computes $z_{off} = f^{s_\mathrm{ID}} f^\alpha = f^{s_\mathrm{ID} + \alpha}$;
\item sets the offline signature as $\sigma_{off} = \{x, y_1, \ldots, y_k, \\z_{off}, g_{\mathrm{ID}}\}$.
\end{enumerate}
\textbf{Online Signing.}
During this phase, the signer computes the followings:
\begin{enumerate}
\item computes  $h = H_1(m, x)$;
\item computes $z = d_0^{s} z_{off}^h =  d_0^sf^{(s_\mathrm{ID} + \alpha)h}$;
\item sets signature as $\sigma_{on} = \{z\}$.
\end{enumerate}
\par
The signature is $\sigma = \{x, y_1, \ldots, y_k, z, g_{\mathrm{ID}}\}$.
\vspace{0.2cm}
\\
\textbf{Verification.} To verify $\sigma = \{x, y_1, \ldots, y_k, z, g_{\mathrm{ID}}\}$ on message $m$ with respect to identity $\mathrm{ID} = \{I_1, \ldots, I_k\}$, the verification algorithm works as follows:
\begin{enumerate}
\item computes $h = H_1(m, x)$ and $f = H_2(g_\mathrm{ID}, \mathrm{ID})$;
\item checks whether the equation holds:
$\hat{e}(g, z) \\= \hat{e}(g_1, x\prod_{j=1}^k y_j^{I_j})\hat{e}(f^h, g_\mathrm{ID}g_1)\prod_{j=1}^k \hat{e}(y_j, h_j)$\\
If so, outputs 1. Otherwise, outputs 0.
\end{enumerate}
\par
Actually, if the signature is valid, there is 
\begin{align*}
&\hat{e}(g, z) = \hat{e}(g, d_0^{s} f^{(s_\mathrm{ID}+\alpha)h})\\
= &\hat{e}(g, (g_2^\alpha \prod^k_{j=1} F_j(I_j)^{r_j})^{s}) \hat{e}(g, f^{s_\mathrm{ID}h}f^{\alpha h})\\
= &\hat{e}(g, g_2^{\alpha s}) \hat{e}(g, \prod^k_{j=1} g_1^{I_jr_js} h_j^{r_js}) \hat{e}(g^{s_\mathrm{ID}}, f^h) \hat{e}(g^\alpha, f^h)\\
= &\hat{e}(g^\alpha, g_2^s) \hat{e}(g, \prod^k_{j=1} g_1^{I_jr_js}) \hat{e}(g, \prod^k_{j=1}  h_j^{r_js}) \hat{e}(f^h, g_\mathrm{ID}g_1)\\
= &\hat{e}(g_1, x) \hat{e}(f^h, g_\mathrm{ID}g_1) \hat{e}(\prod^k_{j=1}d_j^{I_js}, g_1) \prod^k_{j=1}\hat{e}(d_j^{s}, h_j)\\
= &\hat{e}(g_1, x) \hat{e}(g_1, \prod^k_{j=1} y_j^{I_j}) \hat{e}(f^h, g_\mathrm{ID}g_1) \prod^k_{j=1}\hat{e}(y_j, h_j)\\
= &\hat{e}(g_1, x\prod_{j=1}^k y_j^{I_j})\hat{e}(f^h, g_\mathrm{ID}g_1)\prod_{j=1}^k \hat{e}(y_j, h_j)
\end{align*}
Note that, although the user public parameter is carried as a part of the signature, the receiver is not constrained to verify whether the signature is generated by the user by checking the parameter. 
That is, the receiver 
\vspace{0.2cm}
\\
\textbf{Blame.} 
Given $\{\mathrm{ID}, m, \sigma = \{x, y_1, \ldots, y_k, z, g_{\mathrm{ID}}\}\}$, the algorithm requires the user parameter $g'_{\mathrm{ID}}$ with respect to the identity ID from the AP. 
It outputs 1 if and only if $g_{\mathrm{ID}} \neq g'_{\mathrm{ID}}$ and 0 otherwise.

\subsection{Security Proof}
The security of our escrow-free HIBOOS scheme is proved according to the following Lemmas. 
\begin{lemma} \label{lemma-ef-hibs}
If there exists an EF-sID-CMA algorithm $\mathcal{A}$ that has non-negligible advantage against our HIBOOS scheme, 
then there exists an algorithm $\mathcal{B}$ that breaks the CDH assumption with non-negligible advantage.
\end{lemma}
\begin{proof}
We show how to construct algorithm $\mathcal{B}$ to win the \emph{EF-sID-CMA} game against the CDH assumption. 
Given $g, g^a, g^b$, $\mathcal{B}$ interacts with $\mathcal{A}$ as follows:
\vspace{0.1cm}
\\
\textbf{Init.} 
$\mathcal{A}$ selects an identity $\mathrm{ID}^* = \{I^*_1, \ldots, I^*_k\} (k \leqslant \ell)$ which will be used in the challenge phase.  
\vspace{0.1cm}
\\
\textbf{Setup.} 
$\mathcal{B}$ generates a bilinear map $\hat{e}: G_1 \times G_1 \rightarrow G_2$ with $g$ as a generator of $G_1$.
It randomly picks $\alpha_1, \ldots, \alpha_\ell, \gamma \in \mathbb{Z}^*_p$, and sets $g_1 = g^a$, $g_2 = g^\gamma$, $h_j =g^{\alpha_j}$ for $j = 1, \ldots, \ell$.
Function $F_j : \mathbb{Z}_p \rightarrow G_1$ is defined as $F_j(x)=g_1^xh_j =g_1^xg^{\alpha_j}$.
$\mathcal{B}$ also maintains hash oracles $H_1: \{0, 1\}^* \times G_1 \rightarrow \mathbb{Z}_p^*$ and $H_2: G_1 \times \{0, 1\}^* \rightarrow G_1$.
Parameters $\{\hat{e}, g, g_1,  g_2, h_1, \ldots, h_\ell, H_1, H_2\}$ are sent to $\mathcal{A}$, and master secret $\mathrm{MSK} = g_2^\alpha = (g^{\gamma})^a = (g^a)^\gamma$ are kept.
\vspace{0.1cm}
\\
\textbf{Query.} $\mathcal{B}$ answers queries made by $\mathcal{A}$, where $\mathcal{A}$ is allowed to make up to $q_1$ hash queries, $q_2$ key queries and $q_3$ signing queries.
\begin{itemize}
	\item \emph{Hash queries}. 
	$\mathcal{B}$ maintains lists $L_1$ and $L_2$ to store the answers of the $H_1$ and $H_2$ oracle, respectively.
	\begin{itemize}
	\item $\mathcal{A}$ submits the $H_1$ hash query with input $(m, x)$, $\mathcal{B}$ checks the list $L$. 
	If an entry is found, the same answer is returned to $\mathcal{A}$; otherwise, $\mathcal{B}$ randomly picks $h \in \mathbb{Z}_p^*$ and returns it to $\mathcal{A}$.
	$\mathcal{B}$ stores $(m, x, h)$ to the list $L_1$.
	\item $\mathcal{A}$ submits the $H_2$ hash query with input $(g_\mathrm{ID}, \mathrm{ID})$, $\mathcal{B}$ checks the list $L_2$. 
		If an entry for the query is found, the same answer will be returned to $\mathcal{A}$. 
		Otherwise, if $\mathrm{ID} \neq \mathrm{ID}^*$, $\mathcal{B}$ randomly picks $\beta_i \in \mathbb{Z}_p^*$ and sets $f = H_2(g_\mathrm{ID}, \mathrm{ID}) = g^{\beta_i}$; if $\mathrm{ID} = \mathrm{ID}^*$, sets $f = g^b$.
		$\mathcal{B}$ stores $(g_\mathrm{ID}, \mathrm{ID}, \beta_i)$ to the list $L_2$.
	\end{itemize}
	\item \emph{Key queries}. 
	$\mathcal{A}$ submits a key generation query with an identity $\mathrm{ID} = \{I_1, \ldots, I_k\}$, where $\mathrm{ID} \neq \mathrm{ID}^*$. 
	$\mathcal{B}$ maintains a ID-key list and generates a private key for $\mathcal{A}$: 
	\begin{itemize}
		\item if $(\mathrm{ID} = \{I'_1, \ldots, I'_u\}, d_{\mathrm{ID}\mid u})$ is in the list, where $u < k$ and $I'_j = I_j$ for $j = 1, \ldots, u$, $\mathcal{B}$ randomly picks $r_{u+1}, \ldots, r_k \in \mathbb{Z}_p^*$ and sets $d_{\mathrm{ID}} = \{d_0, \ldots, d_u, g^{r_{u+1}}, \ldots, g^{r_k}\}$;
		\item if $(\mathrm{ID} = \{I'_1, \ldots, I'_u\}, d_{\mathrm{ID}\mid u})$ is in the list, where $u \geqslant k$ and $I'_j = I_j$ for $j = 1, \ldots, k$, $\mathcal{B}$ sets $d_{\mathrm{ID}} = \{d_0, \ldots, d_k\}$;
		\item otherwise, $\mathcal{B}$ randomly picks $r_1, \ldots, r_k \in \mathbb{Z}_p^*$ and sets $d_{\mathrm{ID}} = \{g_2^\alpha \prod^k_{j=1}(g_1^{I_j}h_j)^{r_j}, g^{r_1},$ $ \ldots, g^{r_k}\}$. 
	\end{itemize}
	$\mathcal{B}$ also randomly picks element $s_\mathrm{ID} \in \mathbb{Z}_p^*$. 
	It stores $(\mathrm{ID}, d_\mathrm{ID}, s_\mathrm{ID})$ into the list, and returns $d_{\mathrm{ID}}$ as well as $g^{s_\mathrm{ID}}$ to $\mathcal{A}$.
	\item \emph{Signing queries}. 
	$\mathcal{A}$ submits a signing query with $\mathrm{ID} = \{I_1, \ldots, I_k\}$, $g_\mathrm{ID}$ and $m$ as input, where $\mathrm{ID} \neq \mathrm{ID}^*$.
	If $\mathcal{B}$ does not have a private key $d_{\mathrm{ID}}$ with respect to ID, it generates the private key by implementing the \emph{KeyGen} algorithm. 
	$\mathcal{B}$ signs the message $m$ according to the \emph{Offline} and \emph{Online} algorithm, and returns $\mathcal{A}$ the signature.
\end{itemize}
\textbf{Challenge.} 
$\mathcal{A}$ finally outputs $(\mathrm{ID}^*, m^*, \sigma^*)$, where $ \sigma^*=\{x^*, y^*_1, \ldots, y^*_k, z^*, g_\mathrm{ID}^*\}$, such that $\sigma^*$ is a valid signature on $m^*$.
\par
According to the principle of forking lemma \cite{Pointcheval1996security}, $\mathcal{B}$ can replay $\mathcal{A}$ with the same random tape but different choices of $H_1$.
It then obtains two valid signatures $(x^*, y_1^*, \ldots, y_k^*, z^*, g_\mathrm{ID}^*)$ and $(x^*, y^*_1, \ldots, y^*_k, \bar{z}^*, g_\mathrm{ID}^*)$ on message $m^*$ with respect to hash functions $H_1$ and $\bar{H_1}$ having different values $h \neq \bar{h}$ on $(m^*, x^*)$, respectively. 
\par



Thus, there is 
\begin{align*}
z^*/\bar{z^*} &= d_0^sf^{(s_\mathrm{ID}+\alpha)h}/d_0^sf^{(s_\mathrm{ID}+\alpha)\bar{h}}\\
&= f^{(s_\mathrm{ID}+\alpha)(h-\bar{h})}\\
&= f^{(s_\mathrm{ID}(h-\bar{h})}f^{\alpha(h-\bar{h})}\\
&= g^{bs_\mathrm{ID}(h-\bar{h})}g^{ab(h-\bar{h})}
\end{align*}
Therefore, we can compute\\
$g^{ab}=\left(z^*\left(\bar{z^*}g^{bs_\mathrm{ID}(h-\bar{h})^{-1}}\right)\right)^{1/(h-\bar{h})}$.
\par
Therefore, $\mathcal{B}$ can breaks the CDH assumption.
However, Joux and Nguyen have pointed out that the CDH problem in cyclic group is hard \cite{joux2003separating}. 
That is, our HIBOOS scheme is ES-sID-CMA secure.
\end{proof}

\begin{lemma} \label{lemma-eps-hibs}
If there exists an EKA-ID-CMA algorithm $\mathcal{A}$ that has non-negligible advantage against our HIBOOS scheme, 
then there is an algorithm $\mathcal{B}$ that breaks the CDH assumption.
\end{lemma}
\begin{proof}
Supposing \emph{EKA-ID-CMA} algorithm $\mathcal{A}$ can break our HIBS scheme, we show how to construct a PPT algorithm $\mathcal{B}$ to violate the CDH assumption. 
Let $q_1$ be the number of hash queries made by $\mathcal{A}$ to $H_1$, where $H_1$ is treated as a hash oracle.
Note that each hash query to $H_1$ is corresponding to a singing query. 
Let $q_2$ be the number of distinct identities that appear in the signature query.
Given $g, g^a, g^b$, $\mathcal{B}$ interacts with $\mathcal{A}$ as follows:
\vspace{0.2cm}
\\
\textbf{Setup.}
$\mathcal{B}$ generates a bilinear map $\hat{e}: G_1 \times G_1 \rightarrow G_2$ with $g$ as a generator of $G_1$.
It randomly picks $\alpha, \alpha_1, \ldots, \alpha_\ell \in \mathbb{Z}^*_p$, $g_2 \in G_1$, and sets $g_1 = g^\alpha$, $h_j = g^{\alpha_j}$ for $j = 1, \ldots, \ell$.
$\mathcal{B}$ also maintains hash oracles $H_1: \{0, 1\}^* \times G_1 \rightarrow \mathbb{Z}_p^*$ and $H_2: G_1 \times \{0, 1\}^* \rightarrow G_1$.
Parameters $Param = \{\hat{e}, g, g_1,  g_2, h_1, \ldots, h_\ell, H_1, H_2\}$ are sent to $\mathcal{A}$.
%Moreover, for $j = 1, \ldots, \ell$, we define $F_j : \mathbb{Z}_p \rightarrow G_1$ to be the function $F_j(x)=g_1^xh_j =g_1^{x-I^*_j}g^{\alpha_j}$.
$\mathcal{B}$ randomly picks three indexes $\hat{\gamma} \in \{1, \ldots, q_1\}$ and $\hat{\eta} \in \{1, \ldots, q_2\}$. 
$\mathcal{B}$ sets $g_{\mathrm{ID}_{\hat{\eta}}} = g^b$.
\vspace{0.2cm}
\\
\textbf{Query.}
$\mathcal{A}$ submits KeyGen and Signing queries.
\begin{itemize}
	\item \emph{KeyGen queries}. 
	$\mathcal{A}$ submits the $\eta^{th}$ key generation query with an identity $\mathrm{ID} = \{I_1, \ldots, I_k\}$.
	If $\eta = \hat{\eta}$, $\mathcal{B}$ aborts. 
	Otherwise,  $\mathcal{B}$ maintains a ID-key list and generates a private key for $\mathcal{A}$: 
	\begin{itemize}
		\item if $\mathrm{ID} = \{I'_1, \ldots, I'_u\}$ with $d_{\mathrm{ID}\mid u}$ is in the list such that $u < k$ and $I'_j = I_j$ for $j = 1, \ldots, u$, $\mathcal{B}$ randomly picks $r_{u+1}, \ldots, r_k \in \mathbb{Z}_p^*$ and sets $d_{\mathrm{ID}} = \{d_0, \ldots, d_u, g^{r_{u+1}}, \ldots, g^{r_k}\}$;
		\item if $\mathrm{ID} = \{I'_1, \ldots, I'_u\}$ with $d_{\mathrm{ID}\mid u}$ is in the list such that $u \geqslant k$ and $I'_j = I_j$ for $j = 1, \ldots, k$, $\mathcal{B}$ sets $d_{\mathrm{ID}} = \{d_0, \ldots, d_k\}$;
		\item otherwise, $\mathcal{B}$ randomly picks $r_1, \ldots, r_k \in \mathbb{Z}_p^*$ and sets $d_{\mathrm{ID}} = \{g_2^\alpha \prod^k_{j=1}(g_1^{I_j}h_j)^{r_j},$ $ g^{r_1}, \ldots, g^{r_k}\}$. 
	\end{itemize}
	$\mathcal{B}$ also randomly picks $s_\mathrm{ID} \in \mathbb{Z}_p^*$. 
	It stores $(\mathrm{ID}, d_\mathrm{ID}, s_\mathrm{ID})$ into the list, and returns $d_{\mathrm{ID}}$ as well as $g^{s_\mathrm{ID}}$ to $\mathcal{A}$.
	\item \emph{Signing queries}. 
	$\mathcal{A}$ submits an identity $\mathrm{ID} = \{I_1, \ldots, I_k\}$ and a message $m$.
	If $\mathcal{B}$ does not have a private key $d_{\mathrm{ID}}$ with respect to ID, it generates the private key by implementing the algorithm in \emph{KeyGen queries}. 
If $\mathrm{ID} = \mathrm{ID}_{\hat{\eta}}$ and $m = m_{\hat{\gamma}}$, $\mathcal{B}$ aborts.
	Otherwise, $\mathcal{B}$ replies the $\gamma^{th}$ signing query as below.
	\begin{itemize}
		\item If $\gamma \neq \hat{\gamma}$, $\mathcal{B}$ randomly picks $r_\gamma \in \mathbb{Z}^*_p$ and sets $f = H_2(g_\mathrm{ID}, \mathrm{ID}) = g^{r_\gamma}$. 
		\item Otherwise, $\mathcal{B}$ sets $f = g^a$.
	\end{itemize}
	$\mathcal{B}$ randomly picks $s \in \mathbb{Z}_p^*$, and computes $x = g_2^s$, $h = H_1(m, x)$, $y_j = d_j^{sh} (j = 1,\ldots, k)$, $z= d_0^{s} f^{(s_\mathrm{ID}+\alpha)h}$.
	Signature $\sigma = \{x, y_1, \ldots, y_k, z, g_{\mathrm{ID}}\}$ is returned to $\mathcal{A}$.
	\end{itemize}
%The keys and signatures returned to $\mathcal{A}$ are well-formed.
%\vspace{0.2cm}
%\\
\textbf{Challenge.}
With probability $\frac{(q_2-1)(q_1q_2-1)}{q_1q_2^2}$, $\mathcal{B}$ does not abort and $\mathcal{A}$ submits $(\mathrm{ID}^*, m^*, \sigma^*)$ such that there exists a Signature Query of which the return values is $(\mathrm{ID^*}, m, \sigma)$ and that $g_{\mathrm{ID}}^* = g_{\mathrm{ID}}$. 
Further, with probability at least $\frac{1}{q_1q_2}$, $g_{\mathrm{ID}} = g_{\mathrm{ID}_{\hat{\eta}}}$ and $m^* = m_{\hat{\gamma}}$.
Thus, there is 
\begin{align*}
&\hat{e}(g, z^*) \\
=~&\hat{e}(g_1, x^*\prod_{j=1}^k y_j^{*I_j})\hat{e}(f^h, g_{\mathrm{ID}}g_1)\prod_{j=1}^k \hat{e}(y_j^*, h_j)\\
=~&\hat{e}(g^\alpha, x^*\prod_{j=1}^k y_j^{*I_j}) \hat{e}(f^h, g_{\mathrm{ID}}g_1) \prod_{j=1}^k \hat{e}(y_j^*, g^{\alpha_j})\\
=~&\hat{e}(g, \left(x^*\prod_{j=1}^k y_j^{*I_j}\right)^\alpha)\hat{e}(g^{ah}, g^bg^\alpha) \hat{e}(g, \prod_{j=1}^k y_j^{*\alpha_j})\\
=~&\hat{e}(g, \left(x^*\prod_{j=1}^k y_j^{*I_j}\right)^\alpha g^{abh}g^{ah\alpha} \prod_{j=1}^k y_j^{*\alpha_j})\\
=~&\hat{e}(g, \left(x^*g^{ah}\right)^{\alpha}\left(\prod_{j=1}^k y_j^{*(\alpha I_j+\alpha_j)}\right) g^{abh})
\end{align*}
So $\mathcal{B}$ gets $z^* = \left(x^*g^{ah}\right)^{\alpha}\left(\prod_{j=1}^k y_j^{*(\alpha I_j+\alpha_j)}\right) g^{abh}$, and computes 
$$g^{ab} = \left(\frac{z^*}{\left(x^*g^{ah}\right)^{\alpha}\prod_{j=1}^k y_j^{*(\alpha I_j+\alpha_j)}}\right)^{1/h}$$
\vspace{0.2cm}
\\
Therefore, PPT algorithm $\mathcal{B}$ can break the CDH assumption.
\end{proof}

\begin{lemma} \label{lemma-eus-hibs}
If there exists an EUS-ID-CMA algorithm $\mathcal{A}$ that has non-negligible advantage against our HIBOOS scheme, 
then there is an algorithm $\mathcal{B}$ that breaks the CDH assumption.
\end{lemma}
\begin{proof}
Supposing \emph{EUS-ID-CMA} algorithm $\mathcal{A}$ can break our HIBS scheme, we show how to construct a PPT algorithm $\mathcal{B}$ to violate the CDH assumption. 
%Let $q_1$ be the number of hash queries made by $\mathcal{A}$ to $H_1$, where $H_1$ is treated as a hash oracle.
%Let $q_2$ be the number of distinct identities that appear in the signature query.
Given $g, g^a, g^b$, $\mathcal{B}$ interacts with $\mathcal{A}$ in a selective identity game as follows:
\vspace{0.2cm}
\\
\textbf{Setup.}
$\mathcal{B}$ generates a bilinear map $\hat{e}: G_1 \times G_1 \rightarrow G_2$ with $g$ as a generator of $G_1$.
It randomly picks $\alpha, \alpha_1, \ldots, \alpha_\ell \in \mathbb{Z}^*_p$, and sets $g_1 = g^a$, $g_2 = g^b$, $h_j =g^{\alpha_j}$ for $j = 1, \ldots, \ell$.
Function $F_j : \mathbb{Z}_p \rightarrow G_1$ is defined as $F_j(x)=g_1^xh_j =g_1^{x}g^{\alpha_j}$.
$\mathcal{B}$ also maintains hash oracles $H_1: \{0, 1\}^* \times G_1 \rightarrow \mathbb{Z}_p^*$ and $H_2: G_1 \times \{0, 1\}^* \rightarrow G_1$.
Parameters $Param = \{\hat{e}, g, g_1,  g_2, h_1, \ldots, h_\ell, H_1, H_2\}$ are sent to $\mathcal{A}$.
\vspace{0.2cm}
\\
\textbf{Query.}
$\mathcal{B}$ answers queries made by $\mathcal{A}$.
\begin{itemize}
	\item \emph{Hash queries}. 
	$\mathcal{B}$ maintains lists $L_1$ and $L_2$ to store the answer of the $H_1$ oracle and $H_2$ oracle respectively.
	\begin{itemize}
		\item $\mathcal{A}$ submits the $i^{th}$ $H_1$ hash query with input $\{m, x\}$,  $\mathcal{B}$ checks the list $L_1$. 
		If an entry for the query is found, the same answer will be returned to $\mathcal{A}$; otherwise, $\mathcal{B}$ randomly picks $h \in \mathbb{Z}_p^*$ and returns to $\mathcal{A}$.
		$\mathcal{B}$ stores $\{m, x, h\}$ to the list $L_1$;
		\item $\mathcal{A}$ submits the $i^{th}$ $H_2$ hash query with input $\{g_\mathrm{ID}, \mathrm{ID}\}$, $\mathcal{B}$ checks the list $L_2$. 
		If an entry for the query is found, the same answer will be returned to $\mathcal{A}$; otherwise, $\mathcal{B}$ randomly picks $\beta_i \in \mathbb{Z}_p^*$ and sets $f = H_2(g_\mathrm{ID}, \mathrm{ID}) = g^{\beta_i}$. 
		$\mathcal{B}$ stores $\{g_\mathrm{ID}, \mathrm{ID}, \beta_i\}$ to the list $L_2$.
	\end{itemize}
	\item \emph{KeyGen queries}. 
	When $\mathcal{A}$ submits a private key query with input $\mathrm{ID} = \{I_1, \ldots, i_k\}$, $\mathcal{B}$ randomly picks $r_1, \ldots, r_k \in \mathbb{Z}_p^*$ and sets
	\begin{align*}
	&d_0 = g_2^{\frac{-\alpha_k}{I_k}}\prod_{j=1}^{k}F_j(I_j)^{r_j}, ~d_1=g^{r_1}, ~\ldots, \\
	&d_{k-1}=g^{r_{k-1}}, ~d_k=g_2^{\frac{-1}{I_k}}g^{r_k}
	\end{align*}
	Note that, there is $d_k = g_2^{\frac{-1}{I_k}}g^{r_k} = g^{\frac{-b}{I_k}}g^{r_k} = g^{r_k-\frac{b}{I_k}}$,
	\begin{align*}
	g_2^{\frac{-\alpha_k}{I_k}}F_k(I_k)^{r_k} &= g_2^{\frac{-\alpha_k}{I_k}}(g_1^{I_k}g^{\alpha_k})^{r_k} \\
	&= g_2^\alpha(g_1^{I_k}g^{\alpha_k})^{r_k-\frac{b}{I_k}} \\
	&= g_2^\alpha F_k(I_k)^{r_k-\frac{b}{I_k}}
	\end{align*}
	Thus we can get 
	\begin{align*}
	&d_0 = g_2^\alpha \left(\prod_{j=1}^{k-1}F_j(I_j)^{r_j}\right)F_k(I_k)^{r_k-\frac{b}{I_k}}, \\
	&d_1=g^{r_1}, ~\ldots, ~d_{k-1}=g^{r_{k-1}}, ~d_k=g^{r_k-\frac{b}{I_k}}
	\end{align*}
	Therefore, the private key returned to $\mathcal{A}$ is well-formed.\\
	For each identity ID, $\mathcal{B}$ maintains a list $L_3$ to store the user key information.
	It randomly picks $s_\mathrm{ID} \in \mathbb{Z}_p^*$, and stores $(\mathrm{ID}, d_\mathrm{ID}, r_1, \ldots, r_k, s_\mathrm{ID})$ into the list $L_3$. 
	Both the private key $d_{\mathrm{ID}}$ and user secret $s_\mathrm{ID}$ are returned to $\mathcal{A}$.
	\item \emph{Signing queries}. 
	When $\mathcal{A}$ submits a signing query with input $\mathrm{ID} = \{I_1, \ldots, I_k\}$ and $m$.
	If $\mathcal{B}$ does not have a private key $d_{\mathrm{ID}}$ with respect to ID, it generates the private key by implementing the algorithm in \emph{KeyGen queries}. 
	With key $d_\mathrm{ID}$ as well as $s_\mathrm{ID}$, $\mathcal{B}$ replies the signing query as below:
	\begin{itemize}
		\item calculates $g_\mathrm{ID} = g^{s_\mathrm{ID}}$;
		\item queries the hash oracle $H_2$ to obtain $\beta_i$ and sets $f = g^{\beta_i}$;
		\item randomly picks $s \in \mathbb{Z}_p^*$, and sets $x = g_2^s$;
		\item queries the hash oracle $H_1$ to obtain $h$;
		\item sets $y_j = d_j^{s+h}$ for $j = 1,\ldots, k$;
		\item sets $z= d_0^{s + h} f^{s_\mathrm{ID}+\alpha} = d_0^{s + h}g^{\beta_i(s_\mathrm{ID}+\alpha)} = d_0^{s + h}g_\mathrm{ID}^{\beta_i}g_1^{\beta_i}$.
	\end{itemize}
	Signature $\sigma = \{x, y_1, \ldots, y_k, z, g_{\mathrm{ID}}\}$ is returned to $\mathcal{A}$.
\end{itemize}
\textbf{Challenge.} 
$\mathcal{A}$ finally outputs $(\mathrm{ID}^*, m^*, \sigma^*)$, where $ \sigma^*=\{x^*, y^*_1, \ldots, y^*_k, z^*, g^*_\mathrm{ID}\}$ and the private key of $\mathrm{ID}^*$ has been queried during the \emph{Query Phase}. 
$\mathcal{B}$ extracts the private key entry $(\mathrm{ID}^*, d_{\mathrm{ID}^*}, r_1, \ldots, r_k, s_{\mathrm{ID}'})$ from list $L_3$. 
Note that, since $\mathcal{A}$ can break our HIBS scheme against the \emph{EUS-ID-CMA} game, there is $g'_\mathrm{ID} \neq g^*_\mathrm{ID}$, where $g'_\mathrm{ID} = g^{s_{\mathrm{ID}'}}$ is the extracted public secret of $\mathrm{ID}^*$.
\par
Following the principle of forking lemma \cite{Pointcheval1996security}, $\mathcal{B}$ can replay $\mathcal{A}$ with the same random tape but different choices of $H_1$.
It then obtains two valid signatures $(x^*, y_1^*, \ldots, y_k^*, z^*, g_\mathrm{ID}^*)$ and $(x^*, \bar{y^*_1}, \ldots, \bar{y^*_k}, \bar{z^*}, \bar{g^*_\mathrm{ID}})$ on message $m^*$ with respect to hash functions $H_1$ and $\bar{H_1}$ having different values $h \neq \bar{h}$ on $(m^*, x^*)$, respectively. 
For $j = 1, \ldots, k$, since $y^*_j = d_j^{s+h}$ and $\bar{y^*_j} = d_j^{s+\bar{h}}$, we can calculate $(y^*_j / \bar{y^*_j}) = d_j^{h-\bar{h}} = g^{r_j(h-\bar{h})}$. 
Thus, there is 
\begin{align*}
z^*\bar{z^*}^{-1} &= g^{bs_j(s_\mathrm{ID}+\alpha)(h-\bar{h})}\\
&= \left(g^{s_\mathrm{ID}bs_j}g^{bs_j\alpha}\right)^{h-\bar{h}}\\
&= \left(g_\mathrm{ID}^{bs_j}g^{ab}s_j\right)^{h-\bar{h}}\\
&= ???
\end{align*}
%\begin{align*}
%&\hat{e}(g, z^*)/\hat{e}(g,\bar{z^*}) \\
%=&\frac{\hat{e}(g_1, g_2^hx\prod_{j=1}^k y_j^{*I_j})\hat{e}(f, g^*_{\mathrm{ID}}g_1)\prod_{j=1}^k \hat{e}(y^*_j, h_j)}{\hat{e}(g_1, g_2^{\bar{h}}x\prod_{j=1}^k \bar{y^*_j}^{I_j})\hat{e}(f, \bar{g^*_{\mathrm{ID}}}g_1)\prod_{j=1}^k \hat{e}(\bar{y^*_j}, h_j)}\\
%=& \hat{e}(g_1, g_2^{h/\bar{h}}\prod_{j=1}^k (y_j^*/\bar{y_j^*})^{I_j})\frac{\hat{e}(g^{\beta_u}, g^*_{\mathrm{ID}}g_1)}{\hat{e}(g^{\beta_v}, \bar{g^*_{\mathrm{ID}}}g_1)}\prod_{j=1}^k \hat{e}(y_j^*/\bar{y_j^*}, h_j)\\
%=& \hat{e}(g_1, g_2^{h-\bar{h}}\prod_{j=1}^k d_j^{(h-\bar{h})I_j})\hat{e}(g, (g^*_{\mathrm{ID}}g_1)^{\beta_u}(\bar{g^*_{\mathrm{ID}}}g_1)^{-\beta_v})\cdot\\
%&\prod_{j=1}^k \hat{e}(d_j^{h-\bar{h}}, g^{\alpha_j})\\
%=& \hat{e}(g, g^{ab(h-\bar{h})}\prod_{j=1}^k (g^{(h-\bar{h})r_jI_j})^a) \hat{e}(g, g_{\mathrm{ID}}^{*\beta_u}\bar{g^*_{\mathrm{ID}}}^{-\beta_v} g_1^{\beta_u-\beta_v})\cdot\\
%&\hat{e}(\prod_{j=1}^k g^{(h-\bar{h})\alpha_jr_j}, g)\\
%=& \hat{e}(g, g^{ab(h-\bar{h})} g_{\mathrm{ID}}^{*\beta_u}\bar{g^*_{\mathrm{ID}}}^{-\beta_v} g^{a(\beta_u-\beta_v)} \prod_{j=1}^k (g^{aI_j} g^{\alpha_j})^{(h-\bar{h})r_j})\\
%=& \hat{e}(g, z^*\bar{z^*}^{-1})
%\end{align*}
%So $\mathcal{B}$ gets $z^*\bar{z^*}^{-1} = g^{ab(h-\bar{h})} g_{\mathrm{ID}}^{*\beta_u}\bar{g^*_{\mathrm{ID}}}^{-\beta_v} g^{a(\beta_u-\beta_v)}\cdot$ $ \prod_{j=1}^k (g^{aI_j} g^{\alpha_j})^{(h-\bar{h})r_j}$, and computes 
%\begin{align*}
%g^{ab} &= \left(\frac{z^*\bar{z^*}^{-1}}{\left(\prod_{j=1}^k (g^{aI_j} g^{\alpha_j})^{(h-\bar{h})r_j} \right) g_{\mathrm{ID}}^{*\beta_u}\bar{g^*_{\mathrm{ID}}}^{-\beta_v} g^{a(\beta_u-\beta_v)}}\right)^{1/(h-\bar{h})}\\
%&= \left(\frac{z^* (\bar{g^*_{\mathrm{ID}}}g^a)^{\beta_v}}{\bar{z^*}\left(\prod_{j=1}^k (g^{aI_j} g^{\alpha_j})^{(h-\bar{h})r_j} \right) (g_{\mathrm{ID}}^* g^a)^{\beta_u}}\right)^{1/(h-\bar{h})}
%\end{align*}
%%\begin{align*}
%%g^{ab} = \left(\frac{z^* (\bar{g^*_{\mathrm{ID}}}g^a)^{\beta_v}}{\bar{z^*}\left(\prod_{j=1}^k (g^{aI_j} g^{\alpha_j})^{(h-\bar{h})r_j} \right) (g_{\mathrm{ID}}^* g^a)^{\beta_u}}\right)^{1/(h-\bar{h})}
%%\end{align*}
\vspace{0.2cm}
\\
Therefore, PPT algorithm $\mathcal{B}$ can break the CDH assumption.
\end{proof}

\section{Performance Analysis}\label{sec-Performance}
We compare the performance of our HIBOOS scheme with other schemes in Table \ref{table-cp}. 
Signing algorithm computation, verifying algorithm computation are compared between the the schemes.
The $q$ denotes the order of group, $k$ denotes the depth of identity, $E$ represents an exponentiation computation and $P$ represents a pairing computation.\par
The comparing results show that our escrow-free scheme introduces low extra overhead to the primitive HIBS scheme.  
\par

\begin{table*}
\centering
\caption{\label{table-cp} Comparison of different schemes (* The computation overhead depends on the value a random  $\beta \in \mathbb{Z}^*_q$, which up to $\left|q\right| -1$)}
\begin{tabular}{|c|c|c|c|c|}
\hline
Scheme &SHER-IBS \cite{chow2004secure} &K-IBOOS \cite{kar2014provably} &CWS-HIBOOS \cite{anescrowfree2015chen} &Our scheme\\
\hline
\hline
Hierarchical &$\surd$ &$\times$ &$\surd$ &$\surd$\\
\hline
OO model	&$\times$ &$\surd$ &$\surd$ &$\surd$\\
\hline
Escrow-free	&$\times$ &$\times$ &$\times$ &$\surd$\\
\hline
Offline Sign Comp. &$-$ &$(\left| q\right| -1)P$ &$(2k+2)E+2P$ & \\
\hline
Online Sign Comp. &$(k+2)E$ &* &$2E$ & \\
\hline
Verify Comp.	&$(k+1)E+(k+2)P$ &$1E+3P$ &$(k+1)E+(k+2)P$ & \\
\hline
\end{tabular}
\end{table*}

Our HIBOOS scheme extends the SHER-IBS scheme. 
Although introducing heavy computations, especially the two pairing computations, in the offline singing phase, our HIBOOS scheme can achieve high online efficiency that the computation overhead is constant, which is also better than the K-IBOOS scheme.
%Moreover, the K-IBOOS scheme does not support hierarchical identities. 

\section{Related Work} \label{sec-relatedwork}
on OO signature...
\par

on key escrow problem.... 

\section{Conclusion}\label{sec-conclusion}
conclusions.
\par

\ack This research is supported in part by the project of the program of Changjiang Scholars and Innovative Research Team in University (No. IRT1012); Science and Technology Innovative Research Team in Higher Educational Institutions of Hunan Province (‘network technology’); and Hunan Province Natural Science Foundation of China (11JJ7003).

\begin{thebibliography}{10}

\bibitem{campbell2008rise}
A.~T. Campbell, S.~B. Eisenman, N.~D. Lane, E.~Miluzzo, R.~Peterson, H.~Lu,
  X.~Zheng, M.~Musolesi, K.~Fodor, G.-S. Ahn \emph{et~al.}, ``The rise of
  people-centric sensing,'' \emph{Internet Computing, IEEE}, vol.~12, no.~4,
  pp. 12--21, 2008.

\bibitem{cornelius2008anonysense}
C.~Cornelius, A.~Kapadia, D.~Kotz, D.~Peebles, M.~Shin, and N.~Triandopoulos,
  ``Anonysense: privacy-aware people-centric sensing,'' in \emph{Proceedings of
  the 6th international conference on Mobile systems, applications, and
  services}.\hskip 1em plus 0.5em minus 0.4em\relax ACM, 2008, pp. 211--224.

\bibitem{shi2010prisense}
J.~Shi, R.~Zhang, Y.~Liu, and Y.~Zhang, ``Prisense: privacy-preserving data
  aggregation in people-centric urban sensing systems,'' in \emph{INFOCOM, 2010
  Proceedings IEEE}.\hskip 1em plus 0.5em minus 0.4em\relax IEEE, 2010, pp.
  1--9.

\bibitem{puttaswamy2010preserving}
K.~P. Puttaswamy and B.~Y. Zhao, ``Preserving privacy in location-based mobile
  social applications,'' in \emph{Proceedings of the Eleventh Workshop on
  Mobile Computing Systems \& Applications}.\hskip 1em plus 0.5em minus
  0.4em\relax ACM, 2010, pp. 1--6.

\bibitem{johnson2007people}
P.~Johnson, A.~Kapadia, D.~Kotz, N.~Triandopoulos, and N.~Hanover,
  ``People-centric urban sensing: Security challenges for the new paradigm,''
  \emph{Dept. of Computer Science, Dartmouth College. URL http://www. cs.
  dartmouth. edu/\~{} dfk/papers/johnson-metrosec-challenges-tr.
  pdf.--Zugriffsdatum}, vol.~26, p. 2008, 2007.

\bibitem{shamir1985identity}
A.~Shamir, ``Identity-based cryptosystems and signature schemes,'' in
  \emph{Advances in cryptology}.\hskip 1em plus 0.5em minus 0.4em\relax
  Springer, 1985, pp. 47--53.

\bibitem{choon2002identity}
J.~C. Choon and J.~H. Cheon, ``An identity-based signature from gap
  diffie-hellman groups,'' in \emph{Public key cryptography—PKC 2003}.\hskip
  1em plus 0.5em minus 0.4em\relax Springer, 2002, pp. 18--30.

\bibitem{chow2004secure}
S.~S. Chow, L.~C. Hui, S.~M. Yiu, and K.~Chow, ``Secure hierarchical identity
  based signature and its application,'' in \emph{Information and
  Communications Security}.\hskip 1em plus 0.5em minus 0.4em\relax Springer,
  2004, pp. 480--494.

\bibitem{gerbush2012dual}
M.~Gerbush, A.~Lewko, A.~O’Neill, and B.~Waters, ``Dual form signatures: An
  approach for proving security from static assumptions,'' in \emph{Advances in
  Cryptology--ASIACRYPT 2012}.\hskip 1em plus 0.5em minus 0.4em\relax Springer,
  2012, pp. 25--42.

\bibitem{boneh2001identity}
D.~Boneh and M.~Franklin, ``Identity-based encryption from the weil pairing,''
  in \emph{Advances in Cryptology—CRYPTO 2001}.\hskip 1em plus 0.5em minus
  0.4em\relax Springer, 2001, pp. 213--229.

\bibitem{kate2010distributed}
A.~Kate and I.~Goldberg, ``Distributed private-key generators for
  identity-based cryptography,'' in \emph{Security and Cryptography for
  Networks}.\hskip 1em plus 0.5em minus 0.4em\relax Springer, 2010, pp.
  436--453.

\bibitem{cao2011sa}
D.~Cao, X.-F. Wang, F.~Wang, Q.-L. Hu, and J.-S. Su, ``Sa-ibe: A secure and
  accountable identity-based encryption scheme,'' \emph{Dianzi Yu Xinxi
  Xuebao(Journal of Electronics and Information Technology)}, vol.~33, no.~12,
  pp. 2922--2928, 2011.

\bibitem{zhang2012efficient}
Y.~Zhang, J.~K. Liu, X.~Huang, M.~H. Au, and W.~Susilo, ``Efficient escrow-free
  identity-based signature,'' in \emph{Provable Security}.\hskip 1em plus 0.5em
  minus 0.4em\relax Springer, 2012, pp. 161--174.

\bibitem{yuen2010construct}
T.~H. Yuen, W.~Susilo, and Y.~Mu, ``How to construct identity-based signatures
  without the key escrow problem,'' \emph{International Journal of Information
  Security}, vol.~9, no.~4, pp. 297--311, 2010.

\bibitem{chen2015efhibs}
P.~Chen, X.~Wang, S.~Hao, and J.~Su, ``An escrow-free hierarchical
  identity-based signature scheme from composite order bilinear groups [in
  press],'' in \emph{The Fifth International Workshop on Cloud, Wireless and
  e-Commerce Security (CWECS 2015)}.\hskip 1em plus 0.5em minus 0.4em\relax
  IEEE, 2015.

\bibitem{joux2003separating}
A.~Joux and K.~Nguyen, ``Separating decision diffie--hellman from computational
  diffie--hellman in cryptographic groups,'' \emph{Journal of cryptology},
  vol.~16, no.~4, pp. 239--247, 2003.

\bibitem{Pointcheval1996security}
D.~Pointcheval and J.~Stern, ``Security proofs for signature schemes,'' in
  \emph{Advances in Cryptology—EUROCRYPT’96}.\hskip 1em plus 0.5em minus
  0.4em\relax Springer, 1996, pp. 387--398.


\bibitem{even1990line}
S.~Even, O.~Goldreich, and S.~Micali, ``On-line/off-line digital signatures,''
  in \emph{Advances in Cryptology—CRYPTO’89 Proceedings}.\hskip 1em plus
  0.5em minus 0.4em\relax Springer, 1990, pp. 263--275.

\bibitem{liu2011online}
J.~K. Liu, J.~Baek, and J.~Zhou, ``Online/offline identity-based signcryption
  revisited,'' in \emph{Information Security and Cryptology}.\hskip 1em plus
  0.5em minus 0.4em\relax Springer, 2011, pp. 36--51.

\bibitem{yasmin2010authentication}
R.~Yasmin, E.~Ritter, and G.~Wang, ``An authentication framework for wireless
  sensor networks using identity-based signatures,'' in \emph{Computer and
  Information Technology (CIT), 2010 IEEE 10th International Conference
  on}.\hskip 1em plus 0.5em minus 0.4em\relax IEEE, 2010, pp. 882--889.

\bibitem{liu2010efficient}
J.~K. Liu, J.~Baek, J.~Zhou, Y.~Yang, and J.~W. Wong, ``Efficient
  online/offline identity-based signature for wireless sensor network,''
  \emph{International Journal of Information Security}, vol.~9, no.~4, pp.
  287--296, 2010.

\bibitem{kar2014provably}
J.~Kar, ``Provably secure online/off-line identity-based signature scheme for
  wireless sensor network.'' \emph{IJ Network Security}, vol.~16, no.~1, pp.
  29--39, 2014.

\bibitem{yang2013id}
X.-d. YANG, C.-m. LI, T.~XU, and C.-f. WANG, ``Id-based on-line/off-line
  threshold signature scheme without bilinear pairing,'' \emph{Journal on
  Communications}, vol.~8, p. 025, 2013.

  
\bibitem{anescrowfree2015chen}
P.~Chen, X.~Wang, and J.~Su, ``An Escrow-Free Hierarchical Identity-Based Signature Model for Cloud Storage [in press],'' 
\emph{The First International Symposium on Dependability in Sensor, Cloud, and Big Data Systems and Applications (DependSys 2015)}.
Springer, 2015.  

\end{thebibliography}

\end{document}
